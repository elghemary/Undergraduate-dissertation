%%%%%%%%%%%%%%%%%%%%%%%%%%%%%%%%%%%%%%%%%%%%%%%%%%
%%%%		~~~~ Introduction ~~~~
%%%%%%%%%%%%%%%%%%%%%%%%%%%%%%%%%%%%%%%%%%%%%%%%%%


\chapter{Introduction}
\label{chap:intro}
\pagestyle{fancy}

Au 17ème siècle, Leibniz a présenté le symbole de dérivation d'ordre $n$, $D^n y = \frac{d^n y}{dx^n} $ où $n$ est un entier positif. cela poussa l'Hopital à se questionnait sur la possibilité d'avoir $n$ dans $\mathbf{Q}$ ? Malgré son scepticisme initial, l'Hopital a anticipé l'émergence d'applications utiles de cette idée en dissent \say{un paradoxe apparent dont l'on tirera un jour d'utiles conséquences}. Ce fut en 1819 que Lacroix formalise pour la première fois la notion de dérivée d'ordre arbitraire marquant aussi une étape importante dans l'évolution de calcul fractionnaire \cite{FC_ross}.\\

Depuis lors, le calcul fractionnaire a connu un développement remarquable.  Des décennies d'études et de recherches ont fait évoluer cette discipline de sa position initiale de domaine purement théorique à une application pratique et précieuse \cite{FDEs_intro}.\\

L'étude des équations différentielles fractionnaires a été particulièrement importante dans ce contexte, celles-ci est devenues un domaine d'intérêt croissant pour les chercheurs en raison de leur capacité à modéliser de manière précise des phénomènes physiques, informatique et d'ingénierie complexes. Cependant, en raison de leur nature non locale et de leur complexité intrinsèque, la résolution de ces équations présente des défis importants. Par conséquent, la création de méthodes numériques efficaces pour résoudre les EDFs est devenue un domaine de recherche majeur. Ces techniques visent à fournir des solutions approchées mais précises afin d'explorer et de comprendre les propriétés et les comportements des systèmes modélisés par les EDFs.\\

Notre objectif dans ce mémoire est d'explorer les solutions numériques des équations différentielles fractionnaire. Bien que plusieurs méthodes existent pour résoudre les EDFs, cette étude se concentre sur l'application de la méthode de la variation d'Homotopie. Cette méthode a été choisie en raison de ses avantages, notamment sa simplicité, sa flexibilité et son efficacité pour résoudre à la fois les EDFs linéaires et non linéaires.\\

Plus précisément, nous visons à :
\begin{itemize}
    \item Revoir et comprendre les définitions et propriétés fondamentales des concepts clés tels que la fonction Gamma, la fonction Bêta et les fonctions de Mittag-Leffler dans le deuxième chapitre.
    \item Examiner les détails du calcul fractionnaire - les intégrales, Dérivées de Riemann-Liouville et de Caputo dans le troisième chapitre.
    \item Appliquer et évaluer la méthode de Perturbation d'Homotopie dans la résolution des EDFs linéaires et non linéaires dans le quatrième chapitre.
\end{itemize}