
%%%%%%%%%%%%%%%%%%%%%%%%%%%%%%%%%%%%%%%%%%%%%%%%%%
%%%%		~~~~ Conclusion ~~~~
%%%%%%%%%%%%%%%%%%%%%%%%%%%%%%%%%%%%%%%%%%%%%%%%%%


\chapter{Conclusion générale}
\label{chap:Conclusion}
\pagestyle{fancy}

En conclusion, ce travail a présenté une étude détaillée sur une des méthodes numériques pour résoudre les EDF, la perturbation d'homotopie. Nous avons commencé par discuter les fondamentaux mathématiques et expliquer en profondeur les dérivées et intégrales fractionnaires.

Nos résultats numériques, qui comparent les solutions exactes et approximatives, démontrent une précision remarquable et satisfaisante de la méthode de perturbation d'homotopie. En effet, les erreurs calculées sont de l'ordre de 10$^{-6}$, ce qui indique une haute précision des solutions approximatives par rapport aux solutions exactes.

Il convient de souligner l'importance des outils computationnels dans notre étude. Nous avons utilisé Python pour générer les figures des fonctions spéciales et comparer les solutions, tandis que Wolfram Alpha a été d'une aide précieuse pour calculer efficacement nos solutions approximatives. Cela illustre l'importance d'une approche interdisciplinaire combinant mathématiques et informatique dans le domaine de la résolution numérique des EDFs.

Cependant, bien que la méthode de perturbation d'homotopie soit efficace et intuitive, elle est l'une parmi plusieurs approches pour résoudre les EDFs. D'autres méthodes pourraient offrir de meilleurs résultats.
