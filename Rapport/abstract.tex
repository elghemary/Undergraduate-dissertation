\begin{abstract}

\noindent
Les équations différentielles fractionnaires sont reconnues pour leur capacité à modéliser des phénomènes naturels complexes plus précisément que les équations différentielles classiques. Cependant, leur résolution pose souvent un défi majeur en raison de leurs propriétés uniques. La méthode de perturbation d'homotopie est un outil puissant pour traiter ces défis. Dans ce travail, nous proposons une analyse en profondeur de cette méthode.
\end{abstract}
\newpage

\topskip0pt
\vspace*{\fill}

\begin{center}
\textbf{Remerciements}
\end{center}
J'aimerais exprimer ma profonde gratitude à mon superviseur, AIT TOUCHENT Kamal, pour son soutien et ses conseils tout au long de ce PFE. Mes sincères remerciements sont aussi adressés au jury pour le temps dédié à l'évaluation de mon travail. De plus, une reconnaissance chaleureuse est adressée à KHARCHOUF Youssef pour son aide indispensable, surtout concernant la partie de codage. Finalement, ma reconnaissance s'étend à tous ceux qui ont contribué, de près ou de loin, à la réalisation de ce projet. À tous, un sincère merci.
\topskip0pt
\vspace*{\fill}